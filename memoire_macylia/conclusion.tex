\chapter*{Conclusion}
\addcontentsline{toc}{chapter}{Conclusion}


La problématique de décision multicritère est souvent présente dans la vie pratique. Du simple choix d’un achat à la sélection d’une carrière, la question demeure la même : comment faire le bon choix en tenant compte de toutes les contradictions qui existent dans les critères qui participent au processus de décision ? La problématique de décision multicritère se réfère à une prise de décision en présence de plusieurs critères, souvent contradictoires.
\vspace{5mm}

Pour les entreprises, une des décisions la plus importante et critique à prendre consiste dans le choix des salariés les plus méritants d’une promotion ou prime et dont le choix doit se faire de manière juste et équitable pour éviter tout sentiment de mise à l’écart de certains salariés ou de favoritisme pour d’autres. Pour aider à cette prise de décision, une solution possible serait de les classer par ordre du plus méritant au moins méritant et cela en les évaluant sur 5 critères majoritairement jugés importants par les entreprises à savoir « les compétences et les résultats, les appréciations sociales, la pénibilité du travail, l’ancienneté dans l’entreprise, l’assiduité», Pour avoir ce classement, on utilise les méthodes d’analyse multicritère « Somme pondérée, ELECTRE III … » des méthodes qui ont déjà montrées leurs efficacités dans de nombreux domaines. Cependant, il est difficile d’estimer les paramètres de ces méthodes qui, utilisent des poids pour quantifier l’importance de chaque critère dans le processus de décision. De ce fait, quand les alternatives offertes par la méthode ne sont pas conformes aux préférences du décideur, une interaction entre la méthode et ce dernier est indispensable afin d’affiner le processus de décision et par conséquent  le processus n’est pas à 100\% automatisé. Une alternative possible à cela « perspective » serait d’utiliser « Les systèmes autonomes dynamiques», qui sont des systèmes auto adaptatifs, capables d’apprendre le comportement de leur environnement afin de s’auto-configurer et de prendre des décisions optimales, sans avoir recours à une aide externe.

