\chapter{Les différentes méthodes d’attribution de primes et de  promotions existantes}


\section{Introduction}

Généralement dans les entreprises, c’est au cours des réunions annuelles que sont fixées les primes et augmentations de salaires et où chaque responsable d’unité défend les intérêts de ses salariés pour obtenir la meilleure part possible de l’enveloppe salariale de l’entreprise. \\
C’est ainsi qu’autour de la table de réunion, sous l’autorité du directeur du personnel qui fera l’arbitrage, on discute sur des notes et des appréciations faites dans le meilleur des cas avec les mêmes barèmes, mais données par des chefs différents, plus ou moins bienveillants.\\ 
En clair, l’augmentation « promotion » de chaque salarié dépend de ce que son chef dira ou montrera le jour de la réunion le concernant. Et plus l’entreprise est grande plus ce fléau empire.\\
Aujourd’hui, pour moderniser cela et pour une amélioration du rendu des employés, plusieurs méthodes ont vu le jour « le ranking, Le management par objectifs (MBO) … »
\\
\section{Le ranking}
\subsection{Qu'est-ce que le " ranking " ?}

Le ranking, est une pratique managériale qui tend à évaluer puis classer les collaborateurs afin d'éliminer les moins performants, cette technique de notation, qui a vu le jour voilà quelques années déjà, est surtout appliquée dans les pays anglo-saxons.\\ 
En clair, elle consiste à classer les salariés, en fonction de leur performance individuelle, selon une distribution fixée à l’avance (20\% très performants, 10\% peu performants, 70\% efficaces)..\\ 
Le choix de cette distribution s’appuie sur une loi statistique représentée par une courbe de Gauss. Les principes sont simples. Le premier est de ne pas cantonner les évaluations par direction, mais au contraire d’utiliser le levier de la transversalité inter-directions.\\ En somme, ce sont des comités constitués des responsables de chaque direction qui vont évaluer l’ensemble des collaborateurs.
\newpage
\subsection{Les différentes étapes du ranking }
Schématiquement, cela fonctionne en quatre étapes : 
\begin{itemize}
\item Premièrement, la DRH répartit les collaborateurs à évaluer en plusieurs groupes sur la base du poids de leur poste et non selon le titre de la fonction du collaborateur.
\item En second lieu, chaque responsable doit réaliser un entretien avec le collaborateur afin de définir les succès de l’année et ses échecs …
\item La réunion d’évaluation, troisième étape, a pour objectif ultime de classer les membres d’un groupe de collaborateurs de poids de postes comparables par ordre décroissant. Pour ce faire, chaque responsable devra mettre en avant des arguments valables pour placer ses collaborateurs ou ceux des autres, dans les premières places et ainsi de suite jusqu’aux dernières places.
\item Enfin, pour être validé, chaque classement requiert le quorum de l’ensemble des responsables.
\end{itemize}

\subsection{Les avantages et inconvénients du ranking}
Cette méthode présente plusieurs avantages :
\begin{itemize}
\item Elle permet de faire évaluer un collaborateur par d’autres responsables ce qui leur permet de connaître ses performances et facilite la mobilité interne.
\item De plus, les décisions prises requièrent le quorum de l’ensemble des responsables, et lorsqu’elles sont bonnes, elles sont plus flatteuses que celles prises par un seul responsable. Et au contraire, une évaluation moyenne, ou mauvaise, est toujours, plus facile à annoncer, pour un manager lorsqu’il se cache derrière une décision de groupe.
\item Ce genre de réunion permet à chaque responsable de défendre les places de leurs protégés dans les meilleures positions du ranking, en exposant des résultats concrets, qui valent mieux que des croix sur des critères dans le système de notation classique. Il devra lutter contre les arguments et contre-arguments des autres responsables qui recherchent aussi à prendre les premières places.
\item Il y a un budget dédié aux performants et pas de budget dédié par direction et c’est une sacrée différence pour la fiche de paie des personnes sélectionnées comme performants ! Cela permet de rejoindre l’objectif fixé par tout système de notation : attirer, retenir et encourager les plus performants.
\end{itemize}

L’inconvénient majeur de cette méthode c’est que la mise en œuvre d'un mode d'évaluation reposant sur un classement des salariés en catégories en fonction de quotas impératifs fixés à l'avance est illicite.\\

D'ailleurs,quelques tribunaux se sont déjà prononcés sur la licéité de ce procédé au vu de  certaines firmes dont l'augmentation ininterrompue des performances est imposée comme   règle et le ranking  devenant l'instrument de mesure privilégié.

\section{Le management par objectifs (MBO)}
\subsection{Le management par objectifs (MBO)}
Le management par objectifs est une pratique très courante dans les entreprises actuelles, c’est le liant entre l'atteinte des objectifs et donc les résultats, d'une part, et le développement de l'individu et son implication, d'autre part.\\
Le système de management par objectifs « Managment by objectives »« MBO » a été proposé en 1954 par Peter Drucker.\\ 
Il présuppose que les mentalités changent et que les individus ne considèrent plus qu’ils sont là pour exécuter une somme de tâches, mais qu’ils sont là pour atteindre l’objectif de leur poste de travail.\\ 
Il anticipe en cela la théorie de la motivation de Frederick Hertzberg qui établit, en 1959, que pour être motivé au travail, l’être humain a besoin de connaître le rôle de son poste dans la marche de l’entreprise et d’assumer la responsabilité des résultats qu’il doit obtenir.

\subsection{Les différentes étapes du MBO}
Le procédé s’effectue de manière annuelle :
\begin{enumerate}
\item Chaque membre du personnel rencontre son responsable en entretien annuel.
\item Ensemble, ils déclinent l’objectif général du poste en sous objectifs mesurables, réalistes, inscrits dans la durée et classés par ordre de priorité. Chaque objectif doit être doté des moyens nécessaires à son accomplissement.
\item Ils se séparent après accord sur le résultat à atteindre et les indicateurs de réussite( un compte rendu d’entretien entérine cet accord).
\item Le collaborateur exécute son travail en restant maître de ses méthodes. Le cas échéant, il décline ses propres objectifs en objectifs pour ses collaborateurs.
\item Responsable et collaborateur se rencontrent régulièrement pour faire le point, à des dates qu’ils ont préalablement arrêtées. Le responsable apporte son aide en savoir, en expérience, en facilitation, en rectification … Mais l’objectif ne change pas. Seuls sont traités les moyens de franchir les obstacles.
\item Lors de l’entretien annuel suivant, responsable et collaborateur se retrouvent pour mesurer le pourcentage de réalisation des objectifs annuels.
\item Ils décident d’actions susceptibles de corriger l’écart entre l’objectif et le résultat. ça peut-être une adaptation de l’objectif, une modification des moyens, une formation pour le collaborateur ou son équipe, etc.
\item Ils optimisent ainsi chaque objectif, en ajoutent si la situation l’exige, annulent ceux qui n’ont plus lieu d’être (projet achevé ou annulé).

\end{enumerate}

\subsection{Les avantages et inconvénients du MBO}

L’avantage principal de cette démarche du management par objectifs mise avant tout sur une relation gagnant-gagnant entre l'employeur et le collaborateur « collaborateur motivé et employeur  satisfait ».\\
L’inconvénient de cette méthode c’est que l’évaluation des salariés se fait sur la base d’un seul critère  « l’atteinte ou pas de l’objectif fixé ».


\section{Le management en équipe}
\subsection{Qu'est-ce que le management en équipe}
Le management en équipe consiste à évaluer les capacités des salariés en fonction de critères objectifs, indépendants des variations aléatoires des performances personnels.\\
Le principe est de considérer l’entreprise comme un système dont les salariés sont des éléments interdépendants c’est-à-dire on ne mets plus l’accent sur des résultats individuels de chaque salarié, mais plutôt sur les résultats commun à tous les salariés.\\
Shewhart, un grand statisticien, chercheur au département qualité de AT\&T, a mis au point en 1939 une méthode pour étudier et améliorer le processus de toute activité de production de biens ou de services, en identifiant des événements significatifs parmi un grand nombre d'événements qui ne le sont pas. Et cela en appliquant le calcul des probabilités à ce type de processus. Le but de cette approche est de supprimer le gâchis matériel et humain du management ordinaire.\\ 
Ce type de management oblige à traiter l’ensemble des processus de production comme un système organique où il est contre-productif de faire appel à la concurrence entre les fournisseurs de l’entreprise, les départements, les employés, etc. Cette conception s'oppose radicalement à la théorie enseignée dans les business schools et aux pratiques dominantes. 

\subsection{La pratique de la méthode du management en équipe}

Bien qu’une telle procédure ne soit pas imposée par la loi, il est important que les directeurs et les chefs de service s’entretiennent en tête à tête avec leurs subordonnés au moins une fois par an pour faire le bilan de leurs activités. \\
L’entreprise aurait une procédure d’évaluation annuelle et contrairement aux autres méthodes  La grille d’évaluation ne comportera pas de notes de performances, ou en d’autres termes pas de bilan individuel résultats / objectifs comme dans le MBO mais elle comportera plutôt des notes relatives aux aptitudes professionnelles, aux compétences dans le métier et au degré de participation de chaque salarié à l’amélioration globale de l’entreprise. \\
Ainsi,organiser des groupes de travail consacrés à des projets d’amélioration des processus où les salariés sont invités à y prendre part, cela fournira un bon moyen d’évaluer leur degré de participation.\\
Cette méthode permet d’identifier les salariés « hors contrôle » qui représentent généralement, on le sait, moins de 10\% dans chaque catégorie. Par définition, ce sont ceux dont les aptitudes sont en dehors de l’intervalle de contrôle du processus. Il faut donc chercher les causes de ces variations. Quand une aptitude est supérieure à l’intervalle, une étude avec l’intéressé peut conduire à une amélioration du processus. Quand elle est inférieure, une étude permet de découvrir quel est le handicap de l’intéressé.\\ 
Par exemple dans une petite entreprise, un graphique de contrôle avait montré que l’aptitude d’un magasinier était inférieure à l’intervalle de contrôle. L’étude a révélé qu’il avait besoin de lunettes : le problème a été vite réglé. Cette étude peut se faire à l’aide d’outils d’analyse statistique à l’usage des non statisticiens que l’on trouve sur Internet.\\ 
Quand un salarié est hors contrôle vers le haut, il pourra faire l’objet d’une promotion. Celui qui est hors contrôle vers le bas a besoin d’aide, peut-être d’une formation. Si l’écart persiste malgré l’aide qui lui est donnée, il faudra lui  changer de poste ou le licencier.
\newpage
\subsection{Les avantages et inconvénients du management en équipe}
L’avantage de cette méthode c’est qu’elle privilégie la coopération à la compétition et utilise  des moyens rationnels pour réussir économiquement ce qui incite les salariés à travailler en équipe pour atteindre des objectifs communs.\\
Un autre avantage c’est le fait qu’elle permet de ne pas imputer injustement aux acteurs les défaillances de la production, mais de les imputer au système où ils interviennent.\\ 
L’inconvénient de cette méthode c’est que l’absence d’objectif et  de concurrence  peut engendrer une diminution de la motivation de chaque employé à donner le meilleur de lui-même en comptant sur le travail des autres.

\section{conclusion}
Chacune des méthodes d’évaluation citée dans ce chapitre, présente des points positifs et des points négatifs, ce qui m’amène à proposer une solution plus au moins hybride afin d’améliorer ces points négatifs et essayer de préserver les points positifs de ces solutions.\\
Ci-dessous ,nous présenterons la configuration du contexte dans lequel  s'intégrera  la solution proposée.     
 
